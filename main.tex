\documentclass{article}
\usepackage[utf8]{inputenc}

\title{\textbf{Preliminary Research Proposal}\\Investigating the Microscopic Mechanics of Hair-Hydrometer via Atomic Force Microscopy }
\author{Jun-Yan Zhang, Yue-Jian Mo}

\date{August 2019}

\usepackage{natbib}
\usepackage{graphicx}

\begin{document}

\maketitle

\begin{abstract}
    
\end{abstract}

\section{The Problem}
\begin{itemize}
\item What's the microscopic mechanism of the Hair-Hydrometer?
\item How a normal kerotin in human hair change in morphology and its Young's Modulus with the change of environment humidity and temperature? 
\item What's the relationship between the kerotin's mechanical properties and the macroscopic mechanical properties of human hair?
\end{itemize}

\section{Background}
A simple hygrometer can be built stretching a bunch of degreased human hair, invented in 1800 by a swiss physicist and the founder of mountain-climbing, Sou....  Even now the hair-hydrometer is still playing an important role in meteorological oservatories all over the world because its high precision. \\
The stiffness of the hair will change with the humidity, temperature and other factors of the environment and thus the hair length will change under a fixed tension applied at two endings of the hair. The previous work done by our team found out the rough relation between hair length and environment factors in a macroscopic view, and we hypothised a physical model to explain the phenomenon using estimated density of Hydrogen bonds between kerotins, which is the core compounds consisting human hair and the chemical equilibrium of Hydrogen bond breaking under different humidity and temperature (hydrogen bonds provide a part of the hair's stiffness), but the microscopic mechanics is still under debating.

Enlightened by the Atomic Force Microscopic(AFM) imaging for Young's Modulus(YM) of amyloid fibrils \citep{lee2016advances}(they used an advanced AFM mode, Peak Force Quantitative Nanomechanics Measurement (Peak Force QNM) mode, to get the YM data of amyloid fibrils under different conditions (e.g., in different environment, after different incubation time) by nano-indentation), we came up an idea that to make kerotins fibrils as the specimen instead of amyloid fibrils.  \\
We can control the environment easily by setting up the whole instrument in a artificial box allowing us to adjust the humidity and the temperature.

\subsection{Literature review} 

\section{Design}
\subsection{Instrumentation}
\subsection{Method}

\section{Expected Result}
\subsection{Expected Finishing Time}

\bibliographystyle{plain}
\bibliography{references}
\end{document}